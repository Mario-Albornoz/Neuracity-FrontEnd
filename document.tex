\documentclass{article}
\usepackage{amsmath}
\usepackage{amssymb}
\usepackage{graphicx}
\usepackage{booktabs}

\title{Statistics (M134) - Chapter 12 Summary}
\author{}
\date{}

\begin{document}

\maketitle

\section{Introduction}
Chapter 12 of Statistics (M134) focuses on advanced topics in statistical inference, particularly hypothesis testing and confidence intervals for population parameters. This chapter builds on the foundational concepts introduced in earlier chapters and extends them to more complex scenarios.

\section{Hypothesis Testing}
\subsection{Null and Alternative Hypotheses}
In hypothesis testing, we start by defining two competing hypotheses:
\begin{itemize}
    \item \textbf{Null Hypothesis ($H_0$)}: A statement of no effect or no difference.
    \item \textbf{Alternative Hypothesis ($H_a$)}: A statement that contradicts the null hypothesis, indicating the presence of an effect or difference.
\end{itemize}

\subsection{Test Statistics}
A test statistic is a standardized value used to determine the likelihood of the null hypothesis. For a sample mean $\bar{X}$, the test statistic $Z$ is calculated as:
\[
Z = \frac{\bar{X} - \mu_0}{\sigma / \sqrt{n}}
\]
where $\mu_0$ is the hypothesized population mean, $\sigma$ is the population standard deviation, and $n$ is the sample size.

\subsection{P-Values and Significance Levels}
The p-value is the probability of observing a test statistic as extreme as, or more extreme than, the one observed, assuming the null hypothesis is true. If the p-value is less than the significance level $\alpha$, we reject the null hypothesis.

\section{Confidence Intervals}
\subsection{Definition}
A confidence interval provides a range of values within which the population parameter is expected to lie, with a certain level of confidence. For a population mean $\mu$, the confidence interval is given by:
\[
\bar{X} \pm Z_{\alpha/2} \left( \frac{\sigma}{\sqrt{n}} \right)
\]
where $Z_{\alpha/2}$ is the critical value from the standard normal distribution corresponding to the desired confidence level.

\subsection{Interpretation}
A 95\% confidence interval means that if we were to take 100 different samples and compute a confidence interval for each, approximately 95 of them would contain the true population parameter.

\section{Applications}
\subsection{One-Sample Z-Test}
The one-sample Z-test is used to determine whether the sample mean $\bar{X}$ differs significantly from a known population mean $\mu_0$. The test statistic is calculated as:
\[
Z = \frac{\bar{X} - \mu_0}{\sigma / \sqrt{n}}
\]

\subsection{Two-Sample Z-Test}
The two-sample Z-test compares the means of two independent samples. The test statistic is:
\[
Z = \frac{(\bar{X}_1 - \bar{X}_2) - (\mu_1 - \mu_2)}{\sqrt{\frac{\sigma_1^2}{n_1} + \frac{\sigma_2^2}{n_2}}}
\]
where $\bar{X}_1$ and $\bar{X}_2$ are the sample means, $\mu_1$ and $\mu_2$ are the population means, and $n_1$ and $n_2$ are the sample sizes.

\section{Conclusion}
Chapter 12 provides a comprehensive overview of hypothesis testing and confidence intervals, essential tools for making inferences about population parameters. Understanding these concepts is crucial for applying statistical methods in real-world scenarios.

\end{document}