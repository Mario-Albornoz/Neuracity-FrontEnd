\documentclass[12pt]{article}
\usepackage{amsmath}
\usepackage{amssymb}
\usepackage{geometry}
\geometry{a4paper, margin=1in}

\title{Trigonometry (T221) - Chapter 3 Summary}
\author{}
\date{}

\begin{document}

\maketitle

\section*{Chapter 3: Trigonometric Identities and Equations}

\subsection*{3.1 Fundamental Trigonometric Identities}
The fundamental trigonometric identities are the building blocks for simplifying and solving trigonometric expressions and equations. The most important identities include:

\begin{itemize}
    \item \textbf{Pythagorean Identities:}
    \[
    \sin^2\theta + \cos^2\theta = 1
    \]
    \[
    1 + \tan^2\theta = \sec^2\theta
    \]
    \[
    1 + \cot^2\theta = \csc^2\theta
    \]
    
    \item \textbf{Reciprocal Identities:}
    \[
    \sin\theta = \frac{1}{\csc\theta}, \quad \cos\theta = \frac{1}{\sec\theta}, \quad \tan\theta = \frac{1}{\cot\theta}
    \]
    
    \item \textbf{Quotient Identities:}
    \[
    \tan\theta = \frac{\sin\theta}{\cos\theta}, \quad \cot\theta = \frac{\cos\theta}{\sin\theta}
    \]
    
    \item \textbf{Even-Odd Identities:}
    \[
    \sin(-\theta) = -\sin\theta, \quad \cos(-\theta) = \cos\theta, \quad \tan(-\theta) = -\tan\theta
    \]
\end{itemize}

\subsection*{3.2 Verifying Trigonometric Identities}
Verifying trigonometric identities involves proving that one side of an equation is equal to the other side using known identities. The general strategy is to:

\begin{itemize}
    \item Start with the more complex side of the equation.
    \item Use algebraic manipulations and fundamental identities to simplify.
    \item Transform the expression step-by-step until it matches the other side.
\end{itemize}

\subsection*{3.3 Sum and Difference Identities}
The sum and difference identities allow us to express the sine, cosine, and tangent of the sum or difference of two angles in terms of the sines and cosines of the individual angles.

\begin{itemize}
    \item \textbf{Sine of Sum and Difference:}
    \[
    \sin(A \pm B) = \sin A \cos B \pm \cos A \sin B
    \]
    
    \item \textbf{Cosine of Sum and Difference:}
    \[
    \cos(A \pm B) = \cos A \cos B \mp \sin A \sin B
    \]
    
    \item \textbf{Tangent of Sum and Difference:}
    \[
    \tan(A \pm B) = \frac{\tan A \pm \tan B}{1 \mp \tan A \tan B}
    \]
\end{itemize}

\subsection*{3.4 Double-Angle and Half-Angle Identities}
Double-angle and half-angle identities are derived from the sum identities and are useful in simplifying expressions involving trigonometric functions of multiple angles.

\begin{itemize}
    \item \textbf{Double-Angle Identities:}
    \[
    \sin 2\theta = 2 \sin\theta \cos\theta
    \]
    \[
    \cos 2\theta = \cos^2\theta - \sin^2\theta = 2\cos^2\theta - 1 = 1 - 2\sin^2\theta
    \]
    \[
    \tan 2\theta = \frac{2\tan\theta}{1 - \tan^2\theta}
    \]
    
    \item \textbf{Half-Angle Identities:}
    \[
    \sin\left(\frac{\theta}{2}\right) = \pm \sqrt{\frac{1 - \cos\theta}{2}}
    \]
    \[
    \cos\left(\frac{\theta}{2}\right) = \pm \sqrt{\frac{1 + \cos\theta}{2}}
    \]
    \[
    \tan\left(\frac{\theta}{2}\right) = \pm \sqrt{\frac{1 - \cos\theta}{1 + \cos\theta}} = \frac{\sin\theta}{1 + \cos\theta} = \frac{1 - \cos\theta}{\sin\theta}
    \]
\end{itemize}

\subsection*{3.5 Solving Trigonometric Equations}
Solving trigonometric equations involves finding all angles that satisfy the equation. The general approach is:

\begin{itemize}
    \item Use algebraic techniques to isolate the trigonometric function.
    \item Apply inverse trigonometric functions to find the principal solution.
    \item Use the periodicity of trigonometric functions to find all solutions.
\end{itemize}

\subsection*{3.6 Applications of Trigonometric Identities}
Trigonometric identities are widely used in various fields such as physics, engineering, and computer graphics. They are essential for:

\begin{itemize}
    \item Simplifying complex expressions.
    \item Solving trigonometric equations.
    \item Modeling periodic phenomena.
\end{itemize}

\section*{Conclusion}
Chapter 3 of Trigonometry (T221) introduces the fundamental identities and techniques for simplifying and solving trigonometric expressions and equations. Mastery of these concepts is crucial for further study in mathematics and its applications.

\end{document}